% Template for seminar reports
% Computer Vision Group, Visual Computing Institute, RWTH Aachen University
\documentclass[twoside,a4paper,10pt,DIV=12,BCOR=12mm]{scrartcl}
\usepackage[utf8]{inputenc}    % allows special symbols in LaTeX source code
\usepackage[english]{babel}    % select language of your report (use "ngerman" for German)
\usepackage[T1]{fontenc}       % font encoding, important for umlauts
\usepackage{lmodern}           % lmodern font
\usepackage{xcolor}            % use and define colors for text and images
\usepackage{graphicx}          % include images
\usepackage{booktabs}          % pretty tables
\usepackage{caption}           % captions for figures
\usepackage{subcaption}        % captions for subfigures
\usepackage{url}               % for website links
\usepackage{datetime}          % month on title page
\usepackage{hyperref}          % make references clickable
\usepackage{xspace}            % correct spaces after e.g. "e.g."
\usepackage{tikz}              % custom plots in LaTeX
\usepackage{pgfplots}          % custom function graphs in LaTeX
\usepackage{bm}                % bold math symbols
\usepackage{amsmath}           % additional math environments
\usepackage{amssymb}           % additional math symbols (includes amsfonts)
\usepackage{enumitem}          % change distances for enumerations
\usepackage{listings}          % code
\usepackage{lipsum}            % placeholder text

\newdateformat{monthyeardate}{\monthname[\THEMONTH] \THEYEAR}

\setlength{\parindent}{0pt}
\setlength{\parskip}{0pt}
\setlist{nosep}

\makeatletter
\DeclareRobustCommand\onedot{\futurelet\@let@token\@onedot}
\def\@onedot{\ifx\@let@token.\else.\null\fi\xspace}
\def\eg{{e.g}\onedot} \def\Eg{\emph{E.g}\onedot}
\def\ie{{i.e}\onedot} \def\Ie{\emph{I.e}\onedot}
\def\cf{{c.f}\onedot} \def\Cf{\emph{C.f}\onedot}
\def\etc{{etc}\onedot} \def\vs{\emph{vs}\onedot}
\def\wrt{w.r.t\onedot} \def\dof{d.o.f\onedot}
\def\etal{\emph{et al}\onedot}
\makeatother

\overfullrule=1ex

\pgfplotsset{compat=newest}


% =========================================================================

\graphicspath{{pictures/}}
\setcounter{secnumdepth}{3}
\setcounter{tocdepth}{3}

% =========================================================================
\begin{document}

% Template for seminar reports
% Seminar Current Topics in Computer Vision and Machine Learning

\begin{titlepage}
\begin{center}
\ 
\vspace{3.5cm}

\textsf{
RWTH Aachen University \\
Faculty of Mathematics, Computer Science and Natural Sciences\\
Chair of Computer Science 13 (Computer Vision) \\
Prof. Dr. Bastian Leibe
}

\rule{\linewidth}{1pt}

\vspace{1.75cm}
\LARGE
\textbf{Proseminar Report}

\vspace{1.7cm}
\huge
Long Short Term Memory

\vspace{3.0cm}
\Large
Leon Benz\\
\large
Matriculation Number: 445034
\vspace{0.25cm}
\Large
Fynn Jansen\\
\large
Matriculation Number: 467964

\vspace{0.5cm}
\monthyeardate\today

\vspace{1.05cm}
\rule{\linewidth}{1pt}

\vspace{0.5cm}
\textsf{\textbf{
\normalsize
\begin{tabular}{ll}
Advisor:  & name of advisor\\
\end{tabular}
}}
\end{center}

\end{titlepage}


\begin{abstract}
% +++++++++++++++++++++++++
% Insert your Abstract here (one paragraph summary)
% +++++++++++++++++++++++++
\end{abstract}

\tableofcontents
\newpage
% =========================================================================


\section{Introduction}

The introduction outlines the motivation for studying Long Short-Term Memory networks,
highlighting their significance in addressing the limitations of traditional neural network
architectures.
(Approximately 1 page, written by Fynn)

\section{Foundational Concepts}

This section provides a brief introduction to foundational topics necessary
for understanding the LSTM architecture.

\subsection{Neural Networks and Training}

Explains fundamental neural network concepts required to understanding LSTM networks:

\begin{itemize}
\item Perceptron and Multi-layer Perceptrons
\item Feedforward Neural Networks
\item Loss Functions
\item Gradient Descent Optimization
\item Backpropagation Algorithm
\end{itemize}
(Approximately 2 pages, written by Leon)

\subsection{Recurrent Neural Networks}

Discusses the basic idea and mathematics behind recurrent neural networks.
(Approximately 1 page, written by Leon)

\subsection{Limitations of Recurrent Neural Networks}

Identifies key issues such as vanishing and exploding gradients in RNNs,
highlighting the need for architectures capable of maintaining long-term dependencies.
(Approximately 1 page, written by Leon)

\section{LSTM Architecture}

Detailed explanation and exploration of LSTM networks.

\subsection{Origin}

Presents historical context, foundational studies, and key developments leading up to Hochreiter and Schmidhuber's 1997 paper.
(Approximately 1 page, written by Leon)

\subsection{Principle of Operation}

Provides an intuitive explanation of the components of an LSTM cell,
including gates and internal memory mechanisms, followed by the mathematical formalization of these concepts.
(Approximately 3 pages, written by Fynn)

\section{Training and Implementation}

Addresses practical considerations in training LSTM networks, including an example implementation to demonstrate real-world applications.

\subsection{Training an LSTM}

Explains gradient descent and backpropagation algorithms specifically adapted to
LSTM networks, detailing both intuitive concepts and formal mathematical descriptions.
(Approximately 2 pages, written by Leon)

\subsection{Example Implementation}

Presents breakdown of a sample LSTM implementation.
(Approximately 2 pages, written by Fynn)

\section{Applications}

Demonstrates real-world scenarios where LSTM networks excel, explaining why they are suited in these contexts.

\subsection{Natural Language Processing}

(Approximately 0.5 pages, written by Leon)

\subsection{Time Series Forecasting}

(Approximately 0.5 pages, written by Fynn)

\subsection{Music and Audio}

(Approximately 0.5 pages, written by Fynn)

\subsection{Medical}

(Approximately 0.5 pages, written by Leon)

\section{Limitations and Alternatives}

Critically evaluates LSTM networks, outlining scenarios where other architectures might be better suited than LSTM networks.

\subsection{Limitations}

Discusses drawbacks including computational intensity, scalability concerns, and issues with gradient stability.
(Approximately 0.5 pages, written by Fynn)

\subsection{Variants}

Examines notable variants discussing their respective enhancements and capabilities.
(Approximately 1-1.5 pages, written by Leon/Fynn)

\subsection{LSTM vs Transformers}

Provides an analytical comparison between LSTM networks and Transformer architectures, evaluating strengths, weaknesses.
(Approximately 1 page, written by Leon)

\section{Conclusion}

Summarizes the relevance of LSTM networks, reflecting on current applications and potential future directions.
(Approximately 0.5 pages, written by Fynn)


% \bibliographystyle{alpha}
% \bibliography{abbrev,seminar_report}
\end{document}
